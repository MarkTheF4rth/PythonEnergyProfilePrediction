\subsection{Context}\label{subsec:context}
Energy usage is a growing concern of the computing industry\cite{FrontierEnergyUsage,HistoricalComputingEnergyTrends},
with Python being a widely used language\cite{TIOBE}, it is a prime candidate for energy optimisation.

Energy optimisation has incentivised approach in several computing sectors, including but not limited to server hosting
and mobile technology.

Data centres account for a significant portion of global electricity demand\cite{IEADataCentres}, which presents a
promising avenue for reducing costs of long-term running services in an effort to reduce such
costs\cite{KIOCostsOfDataCentre, AssetSpireDataCosts}.

Similarly, mobile devices require limited batteries to operate, leading to a rather obvious incentive to optimise their
energy usage for longer uptime\cite{SmartPhoneFeatures}.

\subsection{Issue}\label{subsec:issue}
Surveys on conventional knowledge on energy consumption show a general lack of understanding for avenues to improving
power saving\cite{EnergyConsumptionKnowledge}, this means that even if developers choose to optimise their code for
energy efficiency, they may not know how to do so.
This issue is exacerbated by the opacity of modern proprietary hardware, and the layers of abstraction that modern
computers operate on - leading to a lack of understanding of the energy costs of specific operations.

These issues are not unique to energy optimisation, and the general approach is to target general abstract concepts
that are known to create inefficiencies, such as memory management, caching, and system calls.

The concept that this report hinges on is the fact that performance profiling is relatively a lot easier, as there
are more readily available system agnostic tools that simply measure the time taken for a specific operation to
execute.
This is in contrast to energy profiling, which requires system level tools or physical hardware to measure the energy
readings.
Developers who want to measure the energy profile of their system must find and implement such tools which require
knowledge of the system that the software is running on, information which may not even be readily accessible in
situations such as automated testing or cloud computing.


In summary, while both performance and energy profiling have their own challenges, performance profiling is more
accessible and easier to implement than energy profiling.

\subsection{Objectives}\label{subsec:objectives}
The objective of this report is to study the relationship of Python performance against its energy usage, with the goal
of leveraging performance profiling to predict the energy profile of a Python application.
To do this, we will have to find patterns that affect performance and energy in consistent ratios, such as specific
system calls, memory interactions, or higher level concepts.
Furthermore, we must test that these patterns scale, as specific concepts involving memory management and caching may
require more advanced approaches to make accurate predictions.

\begin{itemize}
 \item How does the performance of a Python application relate to its energy usage?
 \item What techniques are required to accurately infer the energy profile of a Python application?
\end{itemize}
