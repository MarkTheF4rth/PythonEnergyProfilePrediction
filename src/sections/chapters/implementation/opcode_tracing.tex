As purely looking at CPU time doesn't appear to be a reliable method of predicting energy usage, we will now attempt to
look at the opcodes being executed by each script to see if a conclusion can be drawn between executed opcodes and
energy usage.
We have already noted that opcodes can be retrieved by using the sys module, unfortunately there does not appear to be
a process that can also retrieve the operands provided, which have been suggested to affect energy usage, but possibly
not by a significant amount~\cite{OperandPower}

\subsubsection{Opcode Gathering}
In the exploration of profiling tools~\ref{subsubsec:pytracer} we simply took each opcode and printed it to console.
This is a good start, but fails to gather data in a meaningful way, and will be useless for larger programs that execute
more opcodes.

For testing opcode gathering in this section we will use the nbody algorithm, as it scales easily and has a large number
of operations involved.
The primary problem with any kind of granular tracing is that the operations involved in the trace naturally interfere
with the execution of the script itself, as opcodes increase in amount this becomes a very real problem, as we will see
in following examples.

To begin with, we will test