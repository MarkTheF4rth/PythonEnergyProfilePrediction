The first profiling method we will test is simply Bash time, the assumption to make for this section is that the energy
profile can be accurately predicted by simply knowing the amount of time a process has spent in real, user, and kernel
time.

\subsubsection{Internal Consistency}
The first step in our process is to verify that running an algorithm multiple times remains consistent in terms of the
amount of energy used over the time taken.
From the previous section, it is clear that there is a large amount of variance in time taken for algorithms to complete
- this section seeks to prove that the variance of time predictably scales with the variance of energy usage.
If we can prove that the energy usage of an algorithm is consistent over multiple runs, this would suggest that time
measurements are a required component of predicting energy usage.

\subsubsection{Naive Prediction}
The first step in our process will be to test how reliable simply timing a process is.
As covered in~\ref{subsubsec:bash_time}, bash time provides three values for real, user, and sys time, for the naive
approach we will attempt to take the overall energy usage of the application, and then split it into a ratio of each of
these values - real time encompasses both usr and sys time, so the remaining value for the ration will be the difference
between the summed time in the CPU and the real time (for example, if an application takes 1 second of real time, and
has 0.4 seconds of usr and sys time, the ratio would be 0.2:0.4:0.4).

