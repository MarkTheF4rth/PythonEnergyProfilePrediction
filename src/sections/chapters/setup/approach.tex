It is unlikely that a single ratio can model the relationship between performance and energy, this report will attempt
several methods at extrapolating contextual information of a script to predict its energy usage.
Certain programming concepts such as memory management and caching may require more advanced approaches to make accurate
predictions, such as multithreading, which has positive effects on performance at the cost of negative effects on energy
usage\cite{MultithreadingEnergy}.
The goal of this paper is to produce a sufficiently accurate result with minimal effort.... [FINISH]

To find the most accurate model, this report will attempt to use a variety of forecasting techniques to predict energy
readings from performance readings, then compared against actual energy readings to find the most accurate model.

\subsubsection{Experiment Process}
As even the most sanitised systems are unlikely to have a completely stable power usage, we will need to have multiple
runs of each.
For this report we have chosen to run each experiment 10 times, then average results deemed to not be outliers -
outliers will be identified by unexplained rises in temperature, or large deviation from the mean.
There will be a 10-minute gap for each experiment, this length of time has been chosen simply as a convenient length of
time for plotting results, allowing for faster visual inspection of the data.
